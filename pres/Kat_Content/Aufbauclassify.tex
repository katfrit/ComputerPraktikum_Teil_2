\section{Aufbau von \texttt{classify.py}}
\subsection{Laden von Datensätzen}
\begin{frame}[fragile]{Laden von Datensätzen: Alte Version}
        \begin{lstlisting}[language=Python]
try:
    import csv
    data = []
    with open(args.datasetname, 'r') as f:
        reader = csv.reader(f)
        for row in reader:
            if not row:
                continue
            try:
                label = int(row[0])
                features = [float(v) for v in row[1:]]
                data.append((label, features))
            except ValueError:
                print("Warning: skipping malformed row", row)
    print(f"Dataset loaded successfully. ...
        \end{lstlisting}
\end{frame}

\begin{frame}[fragile]{Laden von Datensätzen: Neue Version}
        \begin{lstlisting}[language=Python]
def load_data(filename):
    data = []
    try:
        with open(filename, 'r') as f:
            for line in f:
                if not line.strip(): continue
                parts = line.split(',')
                data.append((float(parts[0]), list(map(float, parts[1:]))))
    except Exception as e:
        print(f"Error while loading: {e}")
        sys.exit(1)
    return data
\end{lstlisting}
\end{frame}

\subsection{Erstellen vom Interface}
\begin{frame}[fragile]{Erstellen vom Interface}
    \begin{figure}
        \includegraphics[width=0.8\linewidth, keepaspectratio]{pix/help_text.png}
    \end{figure}
\end{frame}

\subsection{Die \texttt{run\_cross\_validation} Funktion}
\begin{frame}{Die \texttt{run\_cross\_validation} Funktion}
    \begin{figure}
        \includegraphics[width=0.9\linewidth]{pix/runcross.png}
    \end{figure}
\end{frame}

\begin{frame}[fragile]{Die \texttt{run\_cross\_validation} Funktion}
    \begin{figure}
    \begin{lstlisting}[language=Python, basicstyle=\tiny\ttfamily]
def run_cross_validation(data, l_folds, K_max, mode):
    n = len(data)
    if mode != 1: random.shuffle(data)
    folds = [data[i::l_folds] for i in range(l_folds)]
    fold_errors = {k: [] for k in range(1, K_max + 1)}
    for i in range(l_folds):
        test_set = folds[i]
        train_set = []
        for j in range(l_folds):
            if i != j: train_set.extend(folds[j])

        tree = BallTree(train_set)
        current_fold_counts = {k: 0 for k in range(1, K_max + 1)}

        for y_true, x_test in test_set:
            neighbors = tree.query(x_test, K_max)
            current_sum = 0
            for idx, label in enumerate(neighbors):
                k = idx + 1
                current_sum += label
                y_pred = 1.0 if current_sum >= 0 else -1.0
                if y_pred != y_true:
                    current_fold_counts[k] += 1

        for k in range(1, K_max + 1):
            fold_errors[k].append(current_fold_counts[k] / len(test_set))
    avg_errors = {k: sum(fold_errors[k]) / l_folds for k in range(1, K_max + 1)}
    best_k = min(avg_errors.items(), key=lambda x: x[1])[0]

    return best_k, avg_errors[best_k], avg_errors, fold_errors, folds
    \end{lstlisting}
    \end{figure}
\end{frame}

\subsection{Die main Funktion}

\begin{frame}{Die main Funktion}
    \begin{figure}
        \includegraphics[width=0.3\linewidth, keepaspectratio]{pix/c0.png}
    \end{figure}
\end{frame}

\begin{frame}{Die main Funktion}
    \begin{figure}
        \includegraphics[width=0.3\linewidth, keepaspectratio]{pix/c1.png}
    \end{figure}
\end{frame}

\begin{frame}{Die main Funktion}
    \begin{figure}
        \includegraphics[width=0.3\linewidth, keepaspectratio]{pix/c2.png}
    \end{figure}
\end{frame}

\begin{frame}{Die main Funktion}
    \begin{figure}
        \includegraphics[width=0.3\linewidth, keepaspectratio]{pix/c3.png}
    \end{figure}
\end{frame}

\begin{frame}{Die main Funktion}
    \begin{figure}
        \includegraphics[width=0.3\linewidth, keepaspectratio]{pix/c4.png}
    \end{figure}
\end{frame}