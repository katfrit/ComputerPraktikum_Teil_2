%----------------------------------------------------------------------------------------
%	DOCUMENT CONFIGURATION 
%----------------------------------------------------------------------------------------

\documentclass{beamer}
%----------------------------------------------------------------------------------------
% BEAMER SETUP (LIGHTWEIGHT)
%----------------------------------------------------------------------------------------

\usetheme[progressbar=frametitle]{metropolis}
\usecolortheme{default}
\setbeamertemplate{navigation symbols}{}
\setbeamertemplate{headline}{}
\setbeamertemplate{footline}{}

\graphicspath{{figures/}}

%----------------------------------------------------------------------------------------
% BASIC PACKAGES
%----------------------------------------------------------------------------------------

\usepackage[english]{babel}
\usepackage[T1]{fontenc}
\usepackage{graphicx}

% Math
\usepackage{amsmath,amssymb,amsfonts}
\usepackage{braket}
\usepackage{cancel}

% Animations
\usepackage{media9}

% Colors
\definecolor{darkgreen}{rgb}{0.0,0.6,0.0}

% Figures
\usepackage{caption}
\usepackage{subcaption}

% Code
\usepackage{listings}
\usepackage{xcolor}

\lstset{
  basicstyle=\ttfamily\small,
  keywordstyle=\color{blue},
  commentstyle=\color{gray},
  stringstyle=\color{orange},
  showstringspaces=false,
  breaklines=true
}

%----------------------------------------------------------------------------------------
% BEAMER BLOCK FONTS
%----------------------------------------------------------------------------------------

\setbeamerfont{block title}{series=\bfseries,size=\small}
\setbeamerfont{block body}{size=\footnotesize}

\setbeamerfont{alertblock title}{series=\bfseries,size=\small}
\setbeamerfont{alertblock body}{size=\footnotesize}

\setbeamerfont{exampleblock title}{series=\bfseries,size=\small}
\setbeamerfont{exampleblock body}{size=\footnotesize}

%----------------------------------------------------------------------------------------
% CUSTOM COMMANDS (KEPT)
%----------------------------------------------------------------------------------------

\newcommand{\cancelblue}[2]{\textcolor{blue}{\cancelto{#1}{#2}}}
\newcommand{\cancelb}[1]{\textcolor{blue}{\cancel{#1}}}
\newcommand{\cancelg}[1]{\textcolor{darkgreen}{\cancel{#1}}}
\newcommand{\cancelr}[1]{\textcolor{red}{\cancel{#1}}}

\newcommand{\overblue}[2]{\textcolor{blue}{\overbrace{#2}^{#1}}}
\newcommand{\underblue}[2]{\textcolor{blue}{\underbrace{#2}_{#1}}}

\newcommand{\blue}[1]{\textcolor{blue}{#1}}
\newcommand{\red}[1]{\textcolor{red}{#1}}
\newcommand{\dgreen}[1]{\textcolor{darkgreen}{#1}}

\newcommand{\ulr}[1]{\textcolor{red}{\underline{#1}}}
\newcommand{\ulg}[1]{\textcolor{darkgreen}{\underline{#1}}}
\newcommand{\ulb}[1]{\textcolor{blue}{\underline{#1}}}


%----------------------------------------------------------------------------------------
%	TITLE PAGE & AUTHOR
%----------------------------------------------------------------------------------------

\title[]{{\bf The K-Nearest Neighbors Algorithm}}

\institute[]{Team 2\\Computerpraktikum Teil 2}

\date[]{\today}

\begin{document}

\begin{frame}
  \titlepage
\end{frame}

%----------------------------------------------------------------------------------------
%	ACTUAL CONTENT
%----------------------------------------------------------------------------------------
% \begin{frame}{Überblick}
% \begin{itemize}
%     \item \textbf{Problemstellung}
%     \item \textbf{Aufbau von classify.py}
%     \item \textbf{Aufbau von ball\_tree.py}
%     \begin{itemize}
%         \item Rekursiv konstruiert
%         \item Iterativ konstruiert
%         \item Splitting Methode
%     \end{itemize}
%     \item \textbf{Plotten}
%     \item \textbf{Speed Ups}
%     \item \textbf{Fehlerreduktionsstrategien}
%     \item \textbf{Lernerfolge} %GitHub %Datenstrukturen
%     \item \textbf{Quellen}
% \end{itemize}
% \end{frame}
\begin{frame}{Inhaltsverzeichnis}
    \tableofcontents
\end{frame}
\section{Problemstellung}
\begin{frame}{Problemstellung}
\begin{itemize}
    \item Aufgabe: Methode zur \textbf{binären Klassifikation} in Python
    \item Gegeben: Datenpuntke mit Labels
    \begin{align*}
        D = \{(y_i, x_i)\}_{i=1}^n,~
        y_i \in \{-1,+1\},~
        x_i \in [-1,+1]^d
    \end{align*}
    \item Lerne einen Klassifikator
    \begin{align*}
        f_D : [-1,+1]^d \rightarrow \{-1,+1\}
    \end{align*}
    \item Ziel: \textbf{Minimierung der Fehlklassifikationsrate} auf unbekannten Testdaten $D'$
    \item Ansatz: \textbf{$k$-nächste-Nachbarn} mit Kreuzvalidierung zur Auswahl von $k^*$ mit Ball-Tree
\end{itemize}
\end{frame}

\section{Aufbau von \texttt{classify.py}}
\subsection{Laden von Datensätzen}
\begin{frame}[fragile]{Laden von Datensätzen: Alte Version}
        \begin{lstlisting}[language=Python]
try:
    import csv
    data = []
    with open(args.datasetname, 'r') as f:
        reader = csv.reader(f)
        for row in reader:
            if not row:
                continue
            try:
                label = int(row[0])
                features = [float(v) for v in row[1:]]
                data.append((label, features))
            except ValueError:
                print("Warning: skipping malformed row", row)
    print(f"Dataset loaded successfully. ...
        \end{lstlisting}
\end{frame}

\begin{frame}[fragile]{Laden von Datensätzen: Neue Version}
        \begin{lstlisting}[language=Python]
def load_data(filename):
    data = []
    try:
        with open(filename, 'r') as f:
            for line in f:
                if not line.strip(): continue
                parts = line.split(',')
                data.append((float(parts[0]), list(map(float, parts[1:]))))
    except Exception as e:
        print(f"Error while loading: {e}")
        sys.exit(1)
    return data
\end{lstlisting}
\end{frame}

\subsection{Erstellen vom Interface}
\begin{frame}[fragile]{Erstellen vom Interface}
    \begin{figure}
        \includegraphics[width=0.8\linewidth, keepaspectratio]{pix/help_text.png}
    \end{figure}
\end{frame}

\subsection{Die \texttt{run\_cross\_validation} Funktion}
\begin{frame}{Die \texttt{run\_cross\_validation} Funktion}
    \begin{figure}
        \includegraphics[width=0.9\linewidth]{pix/runcross.png}
    \end{figure}
\end{frame}

\begin{frame}[fragile]{Die \texttt{run\_cross\_validation} Funktion}
    \begin{figure}
    \begin{lstlisting}[language=Python, basicstyle=\tiny\ttfamily]
def run_cross_validation(data, l_folds, K_max, mode):
    n = len(data)
    if mode != 1: random.shuffle(data)
    folds = [data[i::l_folds] for i in range(l_folds)]
    fold_errors = {k: [] for k in range(1, K_max + 1)}
    for i in range(l_folds):
        test_set = folds[i]
        train_set = []
        for j in range(l_folds):
            if i != j: train_set.extend(folds[j])

        tree = BallTree(train_set)
        current_fold_counts = {k: 0 for k in range(1, K_max + 1)}

        for y_true, x_test in test_set:
            neighbors = tree.query(x_test, K_max)
            current_sum = 0
            for idx, label in enumerate(neighbors):
                k = idx + 1
                current_sum += label
                y_pred = 1.0 if current_sum >= 0 else -1.0
                if y_pred != y_true:
                    current_fold_counts[k] += 1

        for k in range(1, K_max + 1):
            fold_errors[k].append(current_fold_counts[k] / len(test_set))
    avg_errors = {k: sum(fold_errors[k]) / l_folds for k in range(1, K_max + 1)}
    best_k = min(avg_errors.items(), key=lambda x: x[1])[0]

    return best_k, avg_errors[best_k], avg_errors, fold_errors, folds
    \end{lstlisting}
    \end{figure}
\end{frame}

\subsection{Die main Funktion}

\begin{frame}{Die main Funktion}
    \begin{figure}
        \includegraphics[width=0.3\linewidth, keepaspectratio]{pix/c0.png}
    \end{figure}
\end{frame}

\begin{frame}{Die main Funktion}
    \begin{figure}
        \includegraphics[width=0.3\linewidth, keepaspectratio]{pix/c1.png}
    \end{figure}
\end{frame}

\begin{frame}{Die main Funktion}
    \begin{figure}
        \includegraphics[width=0.3\linewidth, keepaspectratio]{pix/c2.png}
    \end{figure}
\end{frame}

\begin{frame}{Die main Funktion}
    \begin{figure}
        \includegraphics[width=0.3\linewidth, keepaspectratio]{pix/c3.png}
    \end{figure}
\end{frame}

\begin{frame}{Die main Funktion}
    \begin{figure}
        \includegraphics[width=0.3\linewidth, keepaspectratio]{pix/c4.png}
    \end{figure}
\end{frame}

\section{Aufbau von \texttt{ball\_tree.py}}

\include{Joern_Content/Balltree_rekursiv}

\section{Fehlerreduktionsstrategien}
\subsection{Optimierung}

\begin{frame}[fragile]{Optimierung: Versuchte Methoden}
\begin{columns}[T,onlytextwidth] % [T] aligns to top
  \begin{column}{0.32\textwidth}
    \begin{itemize}
        \item \textbf{stratified l-fold}
        \item additional distance metrics
        \item higher percision summation
        \item leaf size Variation
    \end{itemize}
  \end{column}
  \hfill
  \begin{column}{0.65\textwidth}
    \begin{figure}[T]
    \begin{lstlisting}[language=Python, basicstyle=\tiny\ttfamily]
from collections import defaultdict
import random
def make_stratified_folds(data, l_folds, seed=42):
    rnd = random.Random(seed)
    buckets = defaultdict(list)
    for y, x in data:
        buckets[y].append((y, x))
    for y in buckets:
        rnd.shuffle(buckets[y])
    folds = [[] for _ in range(l_folds)]
    for y, items in buckets.items():
        for i, item in enumerate(items):
            folds[i % l_folds].append(item)
    return folds
    \end{lstlisting}
    \end{figure}
  \end{column}
\end{columns}
\end{frame}

\begin{frame}[fragile]{Optimierung: Versuchte Methoden}
\begin{columns}[T,onlytextwidth] % [T] aligns to top
  \begin{column}{0.32\textwidth}
    \begin{itemize}
        \item stratified l-fold
        \item \textbf{additional distance metrics}
        \item higher percision summation
        \item leaf size Variation
    \end{itemize}
  \end{column}
  \hfill
  \begin{column}{0.65\textwidth}
    \begin{figure}
    \begin{lstlisting}[language=Python, basicstyle=\tiny\ttfamily]
if self.metric == "l2":
  return sum((x - y) ** 2 for x, y in zip(a, b))
if self.metric == "l1":
  return sum(abs(x - y) for x, y in zip(a, b))
# linf
    return max(abs(x - y) for x, y in zip(a, b))
    \end{lstlisting}
    \end{figure}
  \end{column}
\end{columns}
\end{frame}

\begin{frame}[fragile]{Optimierung: Versuchte Methoden}
\begin{columns}[T,onlytextwidth] % [T] aligns to top
  \begin{column}{0.32\textwidth}
    \begin{itemize}
        \item stratified l-fold
        \item additional distance metrics
        \item \textbf{higher percision summation}
        \item \textbf{leaf size Variation}
    \end{itemize}
  \end{column}
  \hfill
  \begin{column}{0.65\textwidth}
    \begin{figure}
    \begin{lstlisting}[language=Python]
math.fsum(...)
    \end{lstlisting}
    \end{figure}
  \end{column}
\end{columns}
\end{frame}

% \begin{frame}[fragile]{Optimierung: Alte Testphase}
%     \begin{lstlisting}[language=Python]
% for y_true, x_test in dataset_test:
%     total_sum = 0.0
%     for tree in fold_classifiers:
%         neighbors = tree.query(x_test, best_k)
%         label_sum = sum(neighbors)
%         f_Di_k = 1.0 if label_sum >= 0 else -1.0
%         total_sum += f_Di_k
        
%     y_pred = 1.0 if total_sum >= 0 else -1.0
%     predictions.append(y_pred)
%     if y_pred != y_true: test_errors += 1

% best_error_rate = test_errors / len(dataset_test) if dataset_test else 0.0
%     \end{lstlisting}
% \end{frame}

% \begin{frame}[fragile]{Optimierung: Neue Testphase}
%     \begin{lstlisting}[language=Python]
% for y_true, x_test in dataset_test:
%     vote_sum = 0.0
%     for tree in ensemble_trees:
%         neighbors = tree.query(x_test, best_k)
%         pred_i = 1.0 if sum(neighbors) >= 0 else -1.0  # sign(0)=1
%         vote_sum += pred_i
%     y_pred = 1.0 if vote_sum >= 0 else -1.0           # sign(0)=1
%     predictions.append(y_pred)
%     if y_pred != y_true: test_errors += 1
%     test_error_rate = test_errors / len(dataset_test) if dataset_test else 0.0
%     \end{lstlisting}
% \end{frame}
\subsection{Testen}


\section{Lernerfolge}

\begin{frame}{Lernerfolge}
    \begin{itemize}
       \item GitHub Erfahrung
       \item Linux und Terminal Nutzung
       \item Fehlersucheingrenzung und Debugging
       \item Datenstrukturen analysieren
       \item Performance Optimierung
    \end{itemize}
\end{frame}


%----------------------------------------------------------------------------------------
%	REFERENCES
%----------------------------------------------------------------------------------------

% \begin{frame}{References}
%     \nocite{*}
%     \setbeamertemplate{bibliography item}{}
%     \printbibliography
% \end{frame}

\end{document}