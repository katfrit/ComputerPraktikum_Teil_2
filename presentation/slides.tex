%----------------------------------------------------------------------------------------
%	DOCUMENT CONFIGURATION 
%----------------------------------------------------------------------------------------

\documentclass{beamer}
%----------------------------------------------------------------------------------------
% BEAMER SETUP (LIGHTWEIGHT)
%----------------------------------------------------------------------------------------

\usetheme[progressbar=frametitle]{metropolis}
\usecolortheme{default}
\setbeamertemplate{navigation symbols}{}
\setbeamertemplate{headline}{}
\setbeamertemplate{footline}{}

\graphicspath{{figures/}}

%----------------------------------------------------------------------------------------
% BASIC PACKAGES
%----------------------------------------------------------------------------------------

\usepackage[english]{babel}
\usepackage[T1]{fontenc}
\usepackage{graphicx}

% Math
\usepackage{amsmath,amssymb,amsfonts}
\usepackage{braket}
\usepackage{cancel}

% Animations
\usepackage{media9}

% Colors
\definecolor{darkgreen}{rgb}{0.0,0.6,0.0}

% Figures
\usepackage{caption}
\usepackage{subcaption}

% Code
\usepackage{listings}
\usepackage{xcolor}

\lstset{
  basicstyle=\ttfamily\small,
  keywordstyle=\color{blue},
  commentstyle=\color{gray},
  stringstyle=\color{orange},
  showstringspaces=false,
  breaklines=true
}

%----------------------------------------------------------------------------------------
% BEAMER BLOCK FONTS
%----------------------------------------------------------------------------------------

\setbeamerfont{block title}{series=\bfseries,size=\small}
\setbeamerfont{block body}{size=\footnotesize}

\setbeamerfont{alertblock title}{series=\bfseries,size=\small}
\setbeamerfont{alertblock body}{size=\footnotesize}

\setbeamerfont{exampleblock title}{series=\bfseries,size=\small}
\setbeamerfont{exampleblock body}{size=\footnotesize}

%----------------------------------------------------------------------------------------
% CUSTOM COMMANDS (KEPT)
%----------------------------------------------------------------------------------------

\newcommand{\cancelblue}[2]{\textcolor{blue}{\cancelto{#1}{#2}}}
\newcommand{\cancelb}[1]{\textcolor{blue}{\cancel{#1}}}
\newcommand{\cancelg}[1]{\textcolor{darkgreen}{\cancel{#1}}}
\newcommand{\cancelr}[1]{\textcolor{red}{\cancel{#1}}}

\newcommand{\overblue}[2]{\textcolor{blue}{\overbrace{#2}^{#1}}}
\newcommand{\underblue}[2]{\textcolor{blue}{\underbrace{#2}_{#1}}}

\newcommand{\blue}[1]{\textcolor{blue}{#1}}
\newcommand{\red}[1]{\textcolor{red}{#1}}
\newcommand{\dgreen}[1]{\textcolor{darkgreen}{#1}}

\newcommand{\ulr}[1]{\textcolor{red}{\underline{#1}}}
\newcommand{\ulg}[1]{\textcolor{darkgreen}{\underline{#1}}}
\newcommand{\ulb}[1]{\textcolor{blue}{\underline{#1}}}


%----------------------------------------------------------------------------------------
%	TITLE PAGE & AUTHOR
%----------------------------------------------------------------------------------------

\title[]{{\bf The K-Nearest Neighbors Algorithm}}

\author[]{Team 2}

\institute[]{Computerpraktikum Teil 2}

\date[]{\today}

\begin{document}

\begin{frame}
  \titlepage
\end{frame}

%----------------------------------------------------------------------------------------
%	ACTUAL CONTENT
%----------------------------------------------------------------------------------------


\begin{frame}{Presentation Overview}
\begin{itemize}
    \item \textbf{Problem Statement}
    \item \textbf{Code Structure} % müssen noch schauen wie das reinpasst, vllt am Ende als Resultat? Oder aufgesplittet in Stücke throughout the pres
    \item \textbf{Ball Tree}
    \begin{itemize}
        \item Constructed Recursively
        \item Constructed Iteratively
    \end{itemize}
    \item \textbf{Speed Ups}
    \item \textbf{Error Reduction Strategies}
    \item \textbf{References}
\end{itemize}
\end{frame}
\begin{frame}
    \centering
    \vfill
    {\Huge \color{blue}{Problem Statement}}\\[0.8em]
    \rule{0.6\linewidth}{0.8pt}
    \vfill
\end{frame} % Divider: Problem Statement
\begin{frame}{Problem Statement}
\begin{itemize}
    \item Implement a method for \textbf{binary classification} in Python
    \item Given labeled data 
    \begin{align*}
        D = \{(y_i, x_i)\}_{i=1}^n,~
        y_i \in \{-1,1\},~
        x_i \in [-1,1]^d
    \end{align*}
    \item Learn a classifier $f_D : [-1,1]^d \rightarrow \{-1,1\}$
    \item Goal: \textbf{minimize the misclassification rate} on unseen test data $D'$
    \item Approach: \textbf{$k$-nearest neighbors} with cross validation to select $k^*$
\end{itemize}
\end{frame}
\begin{frame}
    \centering
    \vfill
    {\Huge \color{blue}{Code Structure}}\\[0.8em]
    \rule{0.6\linewidth}{0.8pt}
    \vfill
\end{frame} % Divider: Code Structure
\begin{frame}{classify.py}
\begin{itemize}
    \item \textbf{Setup}
    \begin{itemize}
        \item Import standard libraries
        \item Custom \texttt{BallTree} class for nearest neighbor search
    \end{itemize}

    \item \textbf{Data Loading}
    \begin{itemize}
        \item \texttt{load\_data}: reads CSV files and stores samples as $(y, x)$ pairs (label and feature vector)
    \end{itemize}

    \item \textbf{Cross Validation}
    \begin{itemize}
        \item \texttt{run\_cross\_validation}: performs $l$-fold cross validation
        \item Evaluates $k = 1,\dots,K_{\max}$ using cumulative majority voting
        \item Computes fold-wise and averaged validation error rates
    \end{itemize}

    \item \textbf{Training}
    \begin{itemize}
        \item Selection of optimal $k^*$ based on minimal validation error
        \item Construction of an ensemble of BallTrees
    \end{itemize}

    \item \textbf{Testing \& Output}
    \begin{itemize}
        \item Ensemble voting for final predictions on the test set
        \item Writes predictions, logs, and error curves to disk
        \item Optional visualization for 2D datasets
    \end{itemize}
\end{itemize}
\end{frame}
\begin{frame}{ball\_tree.py}
\begin{itemize}
    \item \textbf{Tree Construction}
    \begin{itemize}
        \item Data is recursively partitioned into hyperspherical regions (balls)
        \item Each node stores a center and radius enclosing its data
        \item Leaf nodes store the actual data points
        \item Tree is built iteratively using a stack (no recursion)
    \end{itemize}

    \item \textbf{Distance Optimization}
    \begin{itemize}
        \item Precomputation of point norms $\|x\|^2$ for fast distance evaluation
        \item Uses the identity $\|x-y\|^2 = \|x\|^2 - 2\langle x,y\rangle + \|y\|^2$
    \end{itemize}

    \item \textbf{Query Algorithm}
    \begin{itemize}
        \item Best-first search using a priority queue ordered by distance to node centers
        \item Local and global pruning based on node radius and current worst neighbor
        \item Early termination when no closer neighbors are possible
    \end{itemize}
\end{itemize}
\end{frame}
\begin{frame}
    \centering
    \vfill
    {\Huge \color{blue}{Speed Ups}}\\[0.8em]
    \rule{0.6\linewidth}{0.8pt}
    \vfill
\end{frame} % Divider: Speed Ups
\begin{frame}
    \centering
    \vfill
    {\Huge \color{blue}{Error Reduction Strategies}}\\[0.8em]
    \rule{0.6\linewidth}{0.8pt}
    \vfill
\end{frame} % Divider: Errror Reduction Strategies

% ----------------------------------------------------------------------------------------
% PART JÖRN
% ----------------------------------------------------------------------------------------



\begin{frame}
    \centering
    \vfill
    {\Huge \color{blue}{Ball Tree}}\\[0.8em]
    \rule{0.6\linewidth}{0.8pt}
    \vfill
\end{frame} % Divider: Speed Ups

\begin{frame}{Ball Tree: Conceptual}
    \begin{tikzpicture}[overlay, scale=1.2, yshift=0.0cm, xshift=4.0cm]

        \node at (-0.2,-0.2) {\includegraphics[width=1\linewidth]{balltree.png}};

        %\fill[white] (-4.8,3) rectangle (-2.1,-0.2);
        \onslide<-1>{\fill[white] (-1.45,3) rectangle (1.1,-0.2);}
        \onslide<-4>{\fill[white] (1.4,3) rectangle (5.5,-0.2);}
        \onslide<-5>{\fill[white] (-4.8,-0.2) rectangle (-2.1,-3.2);}
        \onslide<-6>{\fill[white] (-1.45,-0.2) rectangle (1.1,-3.2);}

        \begin{scope}[xshift=3.0cm, yshift=-0.5cm]
            \node<2-> at (-0.45,0.0)  {\Large \textbf{\red{At each node}:}};
            \node<3-> at (0.25,-0.5)  {\large 1. Compute center and radius};
            \node<4-> at (0.2,-1.0) {\large 2. Choose two distant points};
            \node<5-> at (0.33,-1.5)  {\large 3. Split points based on points};
            \node<6-> at (0.44,-2.0)  {\large 4. Repeat until leaf size reached};
            \node<7-> at (0.3,-2.6) {\large\red{ \textbf{Leaf nodes store data points}}};
        \end{scope}
    
   \end{tikzpicture}

\end{frame}

\begin{frame}[fragile]{Ball Tree: Recursive}
\begin{semiverbatim}
\footnotesize
\red{def} \orange{BallTree}:

    \red{def} \purple{__init__}(self, data, leaf_size=1):
        self.\blue{leaf_size} = leaf_size
        self.\blue{points} = data 
        self.\blue{left} = None
        self.\blue{right} = None
        self.\blue{center} = self.\blue{_computer_center}(data)
        self.\blue{radius} = self.\blue{_compute_radius}(data, self.\blue{center}) 

        \onslide<2->{\red{if} \purple{len}(data) > self.leaf_size:
            self.\blue{_split}()}

    \onslide<3->{\red{def} \purple{_split}(self):

        ... \dgreen{# Split points}
            
        self.\blue{left} = \purple{BallTree}(left_points, self.\blue{leaf_size})
        self.\blue{right} = \purple{BallTree}(right_points, self.\blue{leaf_size})
        self.\blue{points} = None}
\end{semiverbatim}
\end{frame}

\begin{frame}[fragile]{Ball Tree: Recursive}
    \begin{tikzpicture}[overlay, scale=1.1, yshift=-2.2cm, xshift=5.7cm]

    % Datapoints
    \foreach \x/\y in {
        0.5/1.2,
        1.0/0.8,
        1.3/1.5,
        1.8/1.0,
        2.2/1.4,
        2.6/0.9,
        3.0/1.3,
        3.4/1.0,
        3.8/1.5,
        4.1/1.1,
        1.5/1.9,
        2.0/-0.3,
        1.2/-0.0,
        3.1/2.1,
        3.5/0.3,
        2.9/0.5
    } {
        \filldraw[black] (\x,\y) circle (2pt);
    }

    % First ball (larger)
    \draw[blue, thick, dashed]
        (1.8,1.1) circle (1.5);
    \node[blue, thick] at (1.8,1.1) {\large $\times$};

    % Second ball (smaller)
    \draw[red, thick, dashed]
        (3.2,1.2) circle (1.2);
    \node[red, thick] at (3.2,1.2) {\large $\times$};

    \filldraw[green] (1.0,1.1) circle (2pt);
    \onslide<3->{\filldraw[green] (2.9,0.7) circle (2pt);}
    \onslide<2->{\filldraw[green] (3.8,1.8) circle (2pt);}

    \end{tikzpicture}
\begin{semiverbatim}
\footnotesize

\red{def} \orange{BallTree}:

    ... 

    \onslide<4->{\red{def} \purple{query}(self, x, k, knn=[]):
        lower_bound = \purple{max}(0.0, \purple{distance}(x, self.\blue{center}) - self.\blue{radius})}
        \onslide<5->{\red{if} \purple{len}(knn) == k \blue{and} lower_bound > knn[0][0]:
            \red{return}}

        \onslide<6->{\red{if} self.\blue{left} \blue{is None and} self.\blue{right} \blue{is None}: 

            ... \dgreen{# Pointsearch in leaf node}}

        \onslide<7->{\red{if} \purple{distance}(x, self.\blue{left.center}) < \purple{distance}(x, self.\blue{right.center}):
            self.left.\blue{query}(x, k, knn)
            self.right.\blue{query}(x, k, knn)
        \red{else}:
            ...  }
\end{semiverbatim}
\end{frame}

\begin{frame}[fragile]{Ball Tree: Recursive Speed}
    \begin{enumerate}
        \item Use \textbf{\red{built-in} functions} (C-optimized). \onslide<4->{{\Huge \dgreen{\checkmark}} }\\For example, instead of list comprehension:
            \begin{semiverbatim}
                \small
\red{def} \purple{vector_sum}(vectors):
    \red{return} [\purple{sum}(components) \blue{for} components \blue{in} \purple{zip}(\blue{*}vectors)]
            \end{semiverbatim}
        \item<2-> \red{\textbf{Avoid defining}} often used functions. \onslide<5->{{\Large \red{\text{\sffamily X}} }}
        \item<3-> Save often used variables \red{\textbf{locally}}. \onslide<6->{{\Large \red{\text{\sffamily X}}} }
    \end{enumerate}
    \vspace{0.7cm}
    \LARGE \onslide<7->{\textbf{Recursive Method not optimal \red{!!!}}}
\end{frame}


% ----------------------------------------------------------------------------------------


%----------------------------------------------------------------------------------------
%	REFERENCES
%----------------------------------------------------------------------------------------

\begin{frame}{References}
    \nocite{*}
    \setbeamertemplate{bibliography item}{}
    
    \printbibliography
\end{frame}

\end{document}
